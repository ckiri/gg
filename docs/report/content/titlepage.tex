\selectlanguage{ngerman}
\begin{titlepage}
    \begin{flushleft}
        Software Engineering -- ERP-Systeme (Frank Mysliwitz)\\

        \huge
        \textbf{KI gestützte Stammdatenprüfung: Gefahrengut}\\
        \vspace{1,5cm}

        \Large
        \textbf{Dominik Agreš} {\small (?)}\\
        \textbf{Chris Kiriakou} {\small (209385)}\\
        \textbf{David Koch} {\small (212824)}\\
        \textbf{Berkan Nur} {\small (207365)}\\
        \vspace{1,5cm}
        
        \large
        \section*{Vorwort}
        In diesem Bericht wird die Vorgehensweise zur Semesteraufgabe beschrieben.
        Das Ziel war es, mithilfe von künstlicher Intelligenz, Fehler in den Stammdaten zu
        finden. Hierfür wird eine Einführung in das Gefahrengut geben. Anschließend werden
        drei Ansätze zur Püfung des Gefahrenguts aufgezeigt. 
        Die zum Projekt geschriebene Software befindet sich in einem öffentlich 
        zugänglichen \href{https://github.com/ckiri/gg}{\textcolor{blue}{Repository}}.\\

        \vspace{2,5cm}

        \today \\

        \vspace{3,5cm}
    \end{flushleft}
\end{titlepage}
